\documentclass[12pt]{article}
\usepackage{amsmath}
\title{Logic, Sets, Number Theory, and Counting}
\date{\today}
\author{Walker Smith}


\begin{document}
\maketitle
\pagebreak

\tableofcontents{}
\pagebreak

%
%
% This documentatoin is used as referrence to topics in the feild of 
% Mathematics.
%
%
\section{Introduction}
\LaTeX{} is 
a 
document
preparation
system
for
the
\TeX{}
typesetting
program.

%
%
%
% Logic
%
%
%
\pagebreak
\section{Propositional Logic}

\subsection{Proposition}
A proposition is a declartive sentence 
that is either true or false.
\begin{equation}
p
\end{equation}

\subsection{Negation of p}
The negation of p 
has the opposite truth value of p.
\begin{equation}
\neg p
\end{equation}

\subsection{Conjunction of p and q}
The conjunction of p and q
is true
when
both p and q are true
and false otherwise.
\begin{equation}
p \wedge q
\end{equation}

\subsection{Disjunction of p and q}
The disjunction of p and q
is false
when
both p and q are false
and true otherwise.
\begin{equation}
p \lor q
\end{equation}

\subsection{Exclusive or of p and q}
The exclusive or of p and q
is true
when
exactly one of p and q are true
and false otherwise.
\begin{equation}
p \oplus q
\end{equation}

\subsection{Conditional statement}
The conditional statement if p then q
is false
when
p is true and q is false
and true otherwise.
\begin{equation}
p \rightarrow q
\end{equation}

\subsection{Biconditional statement}
The biconditional statement p if and only if q
is true
when
p and q have the same truth value
and false otherwise.
\begin{equation}
p \leftrightarrow q
\end{equation}

\subsection{Logical Equivalence}
compound propositions p and q are logically equivalent if 
p if and only if q is a tautology,
that is the compound proposition is true 
no matter the truth values of the propositional variables.

\begin{equation}
p \equiv q
\end{equation}

\subsection{Propositional Function}
Value of a propositional function P at x.
Function defined by it's predicate,P and subject, x
where
x is the subject of the statement and
P refers to a property that the subject has.
\begin{equation}
P(x)
\end{equation}

\subsection{Propositional Multivariable Function}
Value of a propositional function P at the n-tuple $(x_{1},x_{2},\dots,x_{n})$.
\begin{equation}
P(x_{1},x_{2},\dots,x_{n})
\end{equation}

\subsection{Universal quantification}
Universal quantification of P(x)
is true
when
P(x) is true for every x
and false
when
there is an x for which P(x) is false.
\begin{equation}
\forall x P(x)
\end{equation}


\subsection{Existential quanification}
Existential quantification of P(x)
is true
when
there exists an element x in the domain such that P(x)
and false
when
P(x) is false for every x.
\begin{equation}
\exists x P(x)
\end{equation}


\subsection{Quantifaction equivalences}
Statement is true
when
there is an x for which P(x) is false
and
false 
when
P(x) is true for every x.
\begin{equation}
\neg \forall x P(x)\equiv\exists x \neg P(x)
\end{equation}


\subsection{Quantifaction equivalences}
Statement is true
when
for every x P(x) is true
and
false 
when 
there is an x for which P(x) is true.
\begin{equation}
\neg \exists x P(x)\equiv\forall x \neg P(x)
\end{equation}

\subsection{Quantification of Two Variables}
P(x,y)
is true
for every pair x,y
and
false
when there is a pair x,y for which P(x,y) is false.

\begin{subequations}\label{quan}
\begin{align}
\forall x \forall y P(x,y)\label{quan_first}\\
\forall y \forall x P(x,y)\label{quan_second}
\end{align}
\end{subequations}

%
For every x there is a y for which P(x,y) is true.
There is an x such that P(x,y) is false for every y.

\begin{equation}
\forall x \exists y P(x,y)
\end{equation}

%
There is an x for which P(x,y) is true for every y.
For every x there is a y for which P(x,y) is false.
\begin{equation}
\exists x \forall y P(x,y)
\end{equation}

%
There is a pair x,y for which P(x,y) is true.
P(x,y) is false for every pair x,y.

\begin{subequations}\label{quanD}
\begin{align}
\exists x \exists y P(x,y)\label{quanD_first}\\
\exists y \exists x P(x,y)\label{quanD_second}
\end{align}
\end{subequations}


\section{Laws of Logical Equivalence}
\setcounter{equation}{0}
\begin{align*}
Logical Equivalences\\
\end{align*}

\subsection{De Morgan's Laws}

\begin{subequations}\label{deMorgan}
\begin{align}
\neg (p \wedge q) \equiv \neg p \lor \neg q\label{de_first}\\
\neg (p \lor q) \equiv \neg p \wedge \neg q\label{de_second}
\end{align}
\end{subequations}

\subsection{Identity Laws}

\begin{subequations}\label{identity}
\begin{align}
p \wedge T \equiv p\label{id_first}\\
p \lor F \equiv p\label{id_second}
\end{align}
\end{subequations}

\subsection{Domination Laws}

\begin{subequations}\label{domination}
\begin{align}
p \lor T \equiv T\label{dom_first}\\
p \wedge F \equiv F\label{dom_second}
\end{align}
\end{subequations}

\subsection{Idempotent Laws}

\begin{subequations}\label{indempotent}
\begin{align}
p \lor p \equiv p\label{indem_first}\\
p \wedge p \equiv p\label{idem_second}
\end{align}
\end{subequations}

\subsection{Double Negation}

\begin{equation}
\neg (\neg p) \equiv p
\end{equation}

\subsection{Communative Laws}

\begin{subequations}\label{communative}
\begin{align}
p \lor q \equiv q \lor p\label{comm_first}\\
p \wedge p \equiv q \wedge p\label{comm_second}
\end{align}
\end{subequations}

\subsection{Associative Laws}

\begin{subequations}\label{associative}
\begin{align}
(p \lor q) \lor r \equiv p \lor (q \lor r)\label{ass_first}\\
(p \wedge q) \wedge r \equiv p \wedge (q \wedge r)\label{ass_second}
\end{align}
\end{subequations}

\subsection{Distributive Laws}

\begin{subequations}\label{distributive}
\begin{align}
p \lor (q \wedge r) \equiv (p \lor q) \wedge (p \lor r)\label{distr_first}\\
p \wedge (q \lor r) \equiv (p \wedge q) \lor (p \wedge r)\label{distr_second}
\end{align}
\end{subequations}

\subsection{Absorption Laws}

\begin{subequations}\label{absorption}
\begin{align}
p \lor (p \wedge q) \equiv p\label{absorb_first}\\
p \wedge (p \lor q) \equiv p\label{absorb_second}
\end{align}
\end{subequations}

\subsection{Absorption Laws}

\begin{subequations}\label{negation}
\begin{align}
p \lor \neg p\equiv T\label{neg_first}\\
p \wedge \neg p \equiv F\label{neg_second}
\end{align}
\end{subequations}
%
%
%
% Set Theory
%
%
%
\pagebreak
\section{Sets}
\setcounter{equation}{0}

\subsection{Set}
A set is an unordered collection of elements where elements are refered to as members of the set.
\begin{equation}
S
\end{equation}

\subsection{Set Membership}
s is an member of the set S.
\begin{equation}
s \in S
\end{equation}

\subsection{Set Membership}
s is not an member of the set S.
\begin{equation}
s \not \in S
\end{equation}

\subsection{Roster Notation}
Roster notation desribes a set by listing the set's elements
\begin{equation}
S = \{s_1, \ldots, s_n\}
\end{equation}

\subsection{Set Builder Notation}
Set builder notation determines membership an element s in a set S based on a property or properties, P(s)
\begin{equation}
S = \{s | P(s)\}
\end{equation}

\subsection{Null Set}
The null set, $\emptyset$, contains no elements
\begin{equation}
\{ \}
\end{equation}

\subsection{Singleton Set}
The singleton set contains exaclty one element, the null set
\begin{equation}
\{ \emptyset\}
\end{equation}

\subsection{Subset}
The set A is a subset of set B, denoted $A \subseteq B$,
is true 
if all members of A are also members of B
and false 
if there is a single $a \in A$ such that $x \not \in B$
\begin{equation}
\forall x (x \in A \rightarrow x \in B)
\end{equation}

\subsection{Proper Subset}
The set A is a proper subset of a set B, denoted $A \subset B$,
is true 
if A is a subset of B 
and 
there exists an x of B that is not an element of A
\begin{equation}
\forall x (x \in A \rightarrow x \in B) \wedge \exists x (x \in B \wedge x \not \in A)
\end{equation}

\subsection{Set Equality}
Two sets are equal, denoted $A = B$,
if and only if
they have the same elements
\begin{equation}
\forall x(x \in A \leftrightarrow x \in B)
\end{equation}

\subsection{Cardinality}
The set A with n distinct elements, where n is a non negative integer,
has a cardinality or size of n.

\begin{equation}
\left| {A} \right|
\end{equation}

\subsection{Power Set}
Given a set A, the power set of A is the set of all subsets of the set A 

\begin{equation}
\mathcal P \left({S}\right)
\end{equation}

%
%
%
% Number Theory
%
%
%
\pagebreak
\section{Number Theory and Cryptography}
\setcounter{equation}{0}
% Eulers 
If 
n
and
m
are
relatively
prime 
for
some
m,n $\in$ $\mathbb{Z}^+$
\begin{align}
n^{\Phi(m)}\equiv 1\mod{m}
\end{align}


% Eulers 
If 
n
and
m
are
relatively
prime 
for
some
m,n $\in$ $\mathbb{Z}^+$
\begin{align}
n^{\Phi(m)} - 1\equiv 0\mod{m}
\end{align}
\end{document}