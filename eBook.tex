\documentclass[12pt]{article}
\usepackage{amsmath}
\title{Logic, Sets, Number Theory, and Counting}
\date{\today}
\author{Walker Smith}


\begin{document}
\maketitle
\pagebreak

\tableofcontents{}
\pagebreak

%
%
% This documentatoin is used as referrence to topics in the feild of 
% Mathematics.
%
%
\section{Introduction}
\LaTeX{} is 
a 
document
preparation
system
for
the
\TeX{}
typesetting
program.

%
%
%
% Logic
%
%
%
\pagebreak
\section{Propositional Logic}

\subsection{Proposition}
A proposition is a declartive sentence 
that is either true or false.
\begin{equation}
\mbox{p}
\end{equation}

\subsection{Negation of p}
The negation of p 
has the opposite truth value of p.
\begin{equation}
\neg \mbox{p}
\end{equation}

\subsection{Conjunction of p and q}
The conjunction of p and q
is true
when
both p and q are true
and false otherwise.
\begin{equation}
\mbox{p} \wedge \mbox{q}
\end{equation}

\subsection{Disjunction of p and q}
The disjunction of p and q
is false
when
both p and q are false
and true otherwise.
\begin{equation}
\mbox{p} \lor \mbox{q}
\end{equation}

\subsection{Exclusive or of p and q}
The exclusive or of p and q
is true
when
exactly one of p and q are true
and false otherwise.
\begin{equation}
\mbox{p} \oplus \mbox{q}
\end{equation}

\subsection{Conditional statement}
The conditional statement if p then q
is false
when
p is true and q is false
and true otherwise.
\begin{equation}
\mbox{p} \rightarrow \mbox{q}
\end{equation}

\subsection{Converse}
The converse of p $\rightarrow$ q
is the conditional statement

\begin{equation}
\mbox{q} \rightarrow \mbox{p}
\end{equation}

\subsection{Contrapositive}
The contrapositive of p $\rightarrow$ q
is the conditional statement

\begin{equation}
\neg \mbox{q} \rightarrow \neg \mbox{p}
\end{equation}

\subsection{Inverse}
The inverse of p $\rightarrow$ q
is the conditional statement

\begin{equation}
\neg \mbox{p} \rightarrow \neg \mbox{q}
\end{equation}

\subsection{Biconditional statement}
The biconditional statement p if and only if q
is true
when
p and q have the same truth value
and false otherwise.
\begin{equation}
\mbox{p} \leftrightarrow \mbox{q}
\end{equation}

\subsection{Logical Equivalence}
compound propositions p and q are logically equivalent if 
p if and only if q is a tautology,
that is the compound proposition is true 
no matter the truth values of the propositional variables.

\begin{equation}
\mbox{p} \equiv \mbox{q}
\end{equation}

\subsection{Propositional Function}
Value of a propositional function P at x.
Function defined by it's predicate,P and subject, x
where
x is the subject of the statement and
P refers to a property that the subject has.
\begin{equation}
P(x)
\end{equation}

\subsection{Propositional Multivariable Function}
Value of a propositional function P at the n-tuple $(x_{1},x_{2},\dots,x_{n})$.
\begin{equation}
P(x_{1},x_{2},\dots,x_{n})
\end{equation}

\subsection{Universal Quantification}
Universal quantification of P(x)
is true
when
P(x) is true for every x
and false
when
there is an x for which P(x) is false.
\begin{equation}
\forall x P(x)
\end{equation}


\subsection{Existential Quanification}
Existential quantification of P(x)
is true
when
there exists an element x in the domain such that P(x)
and false
when
P(x) is false for every x.
\begin{equation}
\exists x P(x)
\end{equation}

\subsection{Quantifaction Equivalences}
Statement is true
when
there is an x for which P(x) is false
and
false 
when
P(x) is true for every x.

\begin{subequations}\label{no}
\begin{align}
\neg \forall x P(x)\equiv\exists x \neg P(x)\\
\intertext{Statement is true
when
for every x P(x) is true
and
false 
when 
there is an x for which P(x) is true.}
\neg \exists x P(x)\equiv\forall x \neg P(x)
\end{align}
\end{subequations}

\subsection{Quantification of Two Variables}
P(x,y)
is true
for every pair x,y
and
false
when there is a pair x,y for which P(x,y) is false.

\begin{subequations}\label{quan}
\begin{align}
\forall x \forall y P(x,y)\label{quan_first}\\
\forall y \forall x P(x,y)\label{quan_second}
\intertext{For every x there is a y for which P(x,y) is true.
There is an x such that P(x,y) is false for every y.}
\forall x \exists y P(x,y)\\
\intertext{There is an x for which P(x,y) is true for every y.
For every x there is a y for which P(x,y) is false.}
\exists x \forall y P(x,y)\\
\intertext{There is a pair x,y for which P(x,y) is true.
P(x,y) is false for every pair x,y.}
\exists x \exists y P(x,y)\label{quanD_first}\\
\exists y \exists x P(x,y)\label{quanD_second}
\end{align}
\end{subequations}

\pagebreak
\section{Laws of Logical Equivalence}
\setcounter{equation}{0}

\subsection{De Morgan's Laws}

\begin{subequations}\label{deMorgan}
\begin{align}
\neg (p \wedge q) \equiv \neg p \lor \neg q\label{de_first}\\
\neg (p \lor q) \equiv \neg p \wedge \neg q\label{de_second}
\end{align}
\end{subequations}

\subsection{Identity Laws}

\begin{subequations}\label{identity}
\begin{align}
p \wedge T \equiv p\label{id_first}\\
p \lor F \equiv p\label{id_second}
\end{align}
\end{subequations}

\subsection{Domination Laws}

\begin{subequations}\label{domination}
\begin{align}
p \lor T \equiv T\label{dom_first}\\
p \wedge F \equiv F\label{dom_second}
\end{align}
\end{subequations}

\subsection{Idempotent Laws}

\begin{subequations}\label{indempotent}
\begin{align}
p \lor p \equiv p\label{indem_first}\\
p \wedge p \equiv p\label{idem_second}
\end{align}
\end{subequations}

\subsection{Double Negation}

\begin{equation}
\neg (\neg p) \equiv p
\end{equation}

\subsection{Communative Laws}

\begin{subequations}\label{communative}
\begin{align}
p \lor q \equiv q \lor p\label{comm_first}\\
p \wedge p \equiv q \wedge p\label{comm_second}
\end{align}
\end{subequations}

\subsection{Associative Laws}

\begin{subequations}\label{associative}
\begin{align}
(p \lor q) \lor r \equiv p \lor (q \lor r)\label{ass_first}\\
(p \wedge q) \wedge r \equiv p \wedge (q \wedge r)\label{ass_second}
\end{align}
\end{subequations}

\subsection{Distributive Laws}

\begin{subequations}\label{distributive}
\begin{align}
p \lor (q \wedge r) \equiv (p \lor q) \wedge (p \lor r)\label{distr_first}\\
p \wedge (q \lor r) \equiv (p \wedge q) \lor (p \wedge r)\label{distr_second}
\end{align}
\end{subequations}

\subsection{Absorption Laws}

\begin{subequations}\label{absorption}
\begin{align}
p \lor (p \wedge q) \equiv p\label{absorb_first}\\
p \wedge (p \lor q) \equiv p\label{absorb_second}
\end{align}
\end{subequations}

\subsection{Absorption Laws}

\begin{subequations}\label{negation}
\begin{align}
p \lor \neg p\equiv T\label{neg_first}\\
p \wedge \neg p \equiv F\label{neg_second}
\end{align}
\end{subequations}


%
%
%
% Set Theory
%
%
%
\pagebreak
\section{Sets}
\setcounter{equation}{0}

\subsection{Set}
A set is an unordered collection of elements where elements are refered to as members of the set.
\begin{equation}
S
\end{equation}

\subsection{Set Membership}
s is a member of the set S.
\begin{equation}
s \in S
\end{equation}

\subsection{Set Membership}
s is not a member of the set S.
\begin{equation}
s \not \in S
\end{equation}

\subsection{Roster Notation}
Roster notation desribes a set by listing the set's elements
\begin{equation}
S = \{s_1, \ldots, s_n\}
\end{equation}

\subsection{Set Builder Notation}
Set builder notation determines membership an element s in a set S based on a property or properties, P(s)
\begin{equation}
S = \{s\mid\text{P(s)}\}
\end{equation}

\subsection{Null Set}
The null set, $\emptyset$, contains no elements
\begin{equation}
\{ \}
\end{equation}

\subsection{Singleton Set}
The singleton set contains exaclty one element, the null set
\begin{equation}
\{ \emptyset\}
\end{equation}

\subsection{Subset}
The set A is a subset of set B, denoted $A \subseteq B$,
is true 
if all members of A are also members of B
and false 
if there is a single $a \in A$ such that $x \not \in B$
\begin{equation}
\forall x (x \in A \rightarrow x \in B)
\end{equation}

\subsection{Proper Subset}
The set A is a proper subset of a set B, denoted $A \subset B$,
is true 
if A is a subset of B 
and 
there exists an x of B that is not an element of A
\begin{equation}
\forall x (x \in A \rightarrow x \in B) \wedge \exists x (x \in B \wedge x \not \in A)
\end{equation}

\subsection{Set Equality}
Two sets are equal, denoted $A = B$,
if and only if
they have the same elements
\begin{equation}
\forall x(x \in A \leftrightarrow x \in B)
\end{equation}

\subsection{Cardinality}
The set A with n distinct elements, where n is a non negative integer,
has a cardinality or size of n.

\begin{equation}
\left| {A} \right|
\end{equation}

\subsection{Power Set}
Given a set A, the power set of A is the set of all subsets of the set A 

\begin{equation}
\mathcal P \left({S}\right)
\end{equation}

\subsection{Ordered n-tuples}
An ordered collection $(a_1, a_2, \ldots, a_n)$ has $a_1$ as its first element, $a_2$ as its second element, \ldots, and $a_n$ as its last element.
\begin{equation}
(a_1, a_2, \ldots, a_n)
\end{equation}

\subsection{Ordered n-tuples Equality}
Two ordered n-tuples, $(a_1, a_2, \ldots, a_n)$ and $(b_1, b_2, \ldots, b_n)$ are equal
if and only if $a_i = b_i$ for $i = 1,2,\ldots,n$
\begin{equation}
(a_1, a_2, \ldots, a_n) = (b_1, b_2, \ldots, b_n)
\end{equation}

\subsection{Cartesian Poduct}
The cartesian product is the set of all ordered pairs (a,b), where a $\in$ A and b $\in$ B
\begin{equation}
A \times B = \{(a,b)\mid\text{a $\in$ A $\wedge$ b $\in$ B}\}
$$
$$\end{equation}

\subsection{Cartesian Poduct of Many Sets}
The cartesian product of the sets $A_1,A_2,\ldots,A_n$,
denoted $A_1 \times A_2 \times \cdots \times A_n$,
is 
the set of ordered n-tuples ($a_1,a_2,\ldots,a_n$)
where $a_i$ belongs to $A_i$ for $i = 1,2,\ldots,n$
\begin{subequations}
\begin{align*}
A_1 \times A_2 \times \cdots \times A_n =\\ 
\{ (a_1,a_2,\ldots,a_n)\mid\text{ $a_i \in A_i$ for $i = 1,2,\ldots,n$} \}
\end{align*}
\end{subequations}


\subsection{Union}
The of union of set A and B , denoted 
is the set containing members of the set A or B
\begin{equation}
\mbox{A} \cup \mbox{B} = \{s\mid\text{s $\in$ A $\lor$ s $\in$ B}\}$$
$$\end{equation}

\subsection{Intersect}
The of intersect of set A and B 
is the set containing members of sets A and B
\begin{equation}
\mbox{A} \cap \mbox{B} = \{s\mid\text{s $\in$ A $\wedge$ s $\in$ B}\}$$
$$\end{equation}

\subsection{Disjoint Sets}
Two sets are disjoint if their intersect is the set containing no elements
\begin{equation}
\mbox{A} \cap \mbox{B} = \emptyset
\end{equation}

\subsection{Difference}
The of difference of set A and B 
is the set containing members of sets A 
that are members of B
\begin{equation}
\mbox{A} - \mbox{B} = \{ s \in A \mid\text{s $\not \in$ B} \}
\end{equation}

\subsection{Truth Set}
Given a predicate P and domain D, the truth set of P 
is 
the set of elements x in D for which P(x) is true
\begin{equation}
\{ x \in D \mid\text P(x) \}
\end{equation}

\subsection{Universal Quantification Over a Set}
$\forall$ x $\in$ S(P(x)) is shorthand notation that denotes
the universal quatification of P(x) over all the elements in the set S.
Statement is true over domain U 
if and only if
the truth set of P = U
\begin{equation}
\forall x \left( x \in S \rightarrow P(x) \right)
\end{equation}

\subsection{Existential Quantification Over a Set}
$\exists$ x $\in$ S(P(x)) is shorthand notation that denotes
the existential quatification of P(x) over all the elements in the set S
\begin{equation}
\exists x \left( x \in S \rightarrow P(x) \right)
\end{equation}


%
%
%
% Relations
%
%
%
\pagebreak
\section{Relations}
\setcounter{equation}{0}

\subsection{Reflexive}
The relation R on a set A is reflexive 
if a ordered pair (a,a) $\in$ R for every a $\in$ A
\begin{equation}
\forall a \in A((a,a) \in R)
\end{equation}

\subsection{Symmetric}
The relation R on a set A is symmetric 
if the ordered pairs (a,b) $\in$ R 
then (b,a) $\in$ R
\begin{equation}
\forall a \forall b \left((a,b)\in R \rightarrow (b,a)\in R \right)
\end{equation}

\subsection{Anti-Symmetric}
The relation R on a set A is antisymmetric 
if the ordered pairs (a,b) $\in$ R 
and (b,a) $\in$ R 
then a = b
\begin{equation}
\forall a \forall b \left(( (a,b)\in R \wedge (b,a)\in R \rightarrow (a = b) \right)
\end{equation}

\subsection{Transative}
The relation R on a set A is transative 
if the ordered pairs (a,b) $\in$ R 
and (b,c) $\in$ R 
then (a,c) $\in$ R
\begin{equation}
\forall a \forall b \forall c \left(( (a,b)\in R \wedge (b,c)\in R \rightarrow (a,c) \in R \right)
\end{equation}

%
%
%
% Number Theory
%
%
%
\pagebreak
\section{Number Theory and Cryptography}
\setcounter{equation}{0}


\subsection{Coprime}
Integers m and n are relatively prime if 
\begin{align}
\gcd (m,n) = 1
\end{align}

\subsection{Euler's Theorm}
If 
n
and
m
are
relatively
prime 
for
some
m,n $\in$ $\mathbb{Z}^+$

\begin{subequations}
\begin{align}
n^{\Phi(m)}\equiv 1\mod{m}\\
\intertext{ }
n^{\Phi(m)} - 1\equiv 0\mod{m}
\end{align}
\end{subequations}

%
%
%
% Graph Theory
%
%
%
\pagebreak
\section{Graph Theory}
\setcounter{equation}{0}

\end{document}